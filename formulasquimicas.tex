\documentclass{article}
\usepackage[utf8]{inputenc}

\title{Formulas químicas en LaTeX}
\author{Toño Avilix}
\date{-}

\usepackage{geometry}
\geometry{textwidth=8cm}

%Paqueterías Matematicas
\usepackage{amssymb, amsmath, amsbsy} 

%Paqueterías Quimicas 
\usepackage{chemfig}
\usepackage{xymtex}

\begin{document}
\maketitle

Formulas y Reacciones Inorgánicas\\ 
\chemfig{H_2SO_4} \\ 
$CO_2+C \rightleftharpoons 2CO $\\
\schemestart $CO_2+C$ \arrow{<=>} $2CO$  \schemestop\par
\schemestart$CO(g)+2H_2(g)$ \arrow{->[\tiny $Fe_2O_3$][\tiny Catalizador]} $CH_3OH(l)$  
\schemestop\par
\begin{equation*}
    CO(g)+2H_2(g) \xrightarrow[Catalizador]{Fe_2O_3} CH_3OH(l)
\end{equation*}
Formulas Química Orgánica \\
\chemfig{O=H}\\
\\Ángulos absolutos \\
\chemfig{A-[:45]B-[:-25]C}
\\Ángulos Relativos \\
\chemfig{A-[::45]B-[::-25]C}
\\Ángulos Predefinidos \\
\chemfig{A-[1]B-[7]C}\\
\\Ramas Ejemplo 1 \\
\chemfig{A-B(-[1]W-X)-C} \\
\\Ramas Ejemplo 2 \\
\chemfig{A-B(-[1]W-X)(-[6]Y-[7]Z)-C}\\
\\Ramas Ejemplo 3 \\
\chemfig{A-B([:60]-D-E)([::-30,1.5]-X-Y)-C}\\
\\Ramas Ejemplo 4 \\
\chemfig{A-B([1]-X([2]-Z)-Y)(-[7]D)-C}\\
\\ \\ \\ Ramas Ejemplo 5 \\
\chemfig{A-B(-[1]W-X?)(-[7]Y-Z)-C?} 
\\ \\ \\Ramas Ejemplo 6 \\
\chemfig{A?[a,2]-B(-[1]W?[a,2]-X?[b])(-[7]Y-Z)-C?[b]} 
\\ \\ Unión de grupo de Atomos \\
\chemfig{ABCD-[:75]EFG}\quad
    \chemfig{ABCD-[1]EFG}
\\ \\
\chemfig{ABCD-[:100]EFG}\quad    
    \chemfig{ABCD-[5]EFG}    
\\ \\
Metano\\
\chemfig{C(-[:0]H)(-[:90]H)(-[:180]H)(-[:270]H)}
\\ \\ Acetaldehído\\
\chemfig{H-C(-[2]H)(-[6]H)-C(-[7]H)=[1]O} \\ \\ 
\\ \\ diagrama esquelético \\
\chemfig{-[:30]-[:-30]-[:30]}
\\2-amino-4-oxohexanoico acido \\
\chemfig{[:30]--[:-30](=[6]O)--[:-30](-[6]NH_2)-(=[2]O)-[:-30]OH}
\\ Ejemplos con anillos \\
\chemfig{A*6(-B=C-D=E-F=)}
\\ \\
\chemfig{*6(=*5(-(=O)-O-(=O)-)-=-=-)}
\\ \\ Notación de Lewis \\ \\ \\
\chemfig{H-[:52.24]\charge{45=\:,135=\:}{O}-[::-104.48]H}
\\ \\ \\ Reacciones orgánicas \\
\schemestart
\chemname{\chemfig{R-C(-[:-30]OH)=[:30]O}}{Ácido Carboxilico} 
\+
\chemname{\chemfig{R’OH}}{Alcohol} 
\arrow{->} 
\chemname{\chemfig{R-C(-[:-30]OR’)=[:30]O}}{Ester} 
\+
\chemname{\chemfig{H_2O}}{Agua} 
\schemestop
\\ \\ Configuración de parámetros \\ 
\chemfig[atom sep=2em]{A-B}\par 
\chemfig[atom sep=50pt]{A-B} \\
\chemfig[bond offset=0pt]{A-B}\par
\chemfig[bond offset=5pt]{A-B}
\\ \\ Proyección de Haworth \\
\chemfig[cram width=2pt]{HO-[2,0.5,2]?<[7,0.7](-[2,0.5]OH)-[,,,,
line width=2pt](-[6,0.5]OH)>[1,0.7](-[6,0.5]OH)-[3,0.7]
O-[4]?(-[2,0.3]-[3,0.5]OH)} \\
 Ejemplos con XyMTeX \\ 
 \cyclohexanev{}
\cyclohexanev{1D==O;3==Cl} \\
\cholestane{3B==HO}
\end{document}