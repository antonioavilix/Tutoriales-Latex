%%%%%%%%%%%%%%%%%%%%%%%%%%%%%%%%%%%%%%%%
%%% Expresiones Matematicas en LaTeX %%%
%%%%%%%%%%%%%%%%%%%%%%%%%%%%%%%%%%%%%%%%

%Author: Antonio Avilix
%Year:2020

%Preambulo
\documentclass[11pt, letterpaper]{article}
\usepackage[utf8]{inputenc}
\usepackage[spanish]{babel}
\usepackage{amsmath, amssymb, amsfonts} %Paqueterias para matematicas
\spanishdecimal{.}    %Uso de punto Decima
\usepackage{esvect}   %Paquete para vectores

%Título, Autor y Fecha
\title{Expresiones Matematicas}
\author{Yo}
\date{2020}

\begin{document}
\maketitle

%---------Potencias, subíndices y superíndices----------------
\section{Potencias, subíndices y superíndices}
$x^p$ $a^{2x}$ $d_2$ $a_{k+a}$ $\sum_{k=1}^N 1+a_{k}$ \\
$\displaystyle \sum_{k=1}^N 1+a_{k}$\\
$9.1$
%-----------------Fraciones----------------------
\section{Fracciones}
$\frac{x+1}{x-1}$ \\ \\
$\displaystyle \frac{x+1}{x-1} $ \\ \\ \\ 
${{x+1} \over 2} \over x-1 $ \\ \\
$\displaystyle {\left(1+ \frac{1}{x} \right)^n}$ \\ \\
$x+1 \atop x-1$ \\ \\
$x+1 \above 2pt x-1$ \\ \\
$x+1 \brack x-1 $ \\
$\displaystyle{a \stackrel{f}{\rightarrow}b}$ \\ \\
$\displaystyle {a \choose b}$ \\
$\displaystyle\binom{n}{k}=\displaystyle\frac{n!}{k!(n-k)!}$
%-----------------Símbolos----------------------
\section{simbolos}
$\alpha + \gamma = \Lambda$ \\
$\sqrt{x+1}$ $\sqrt[3]{3x^4}$ \\
$\Bar{p}$ $\vec{p}$ $\vv{v\times w}$
%-----------------Entornos de Ecuaciones----------------------
\section{Formato de ecuaciones}
Ecuacion \ref{ec:logaritmos} es un ejemplo de propiedad de los logaritmos
\begin{equation}\label{ec:logaritmos}
    log(xy)=logx+logy
\end{equation}
Ejemplo de integral.
\begin{equation}
    \iint\limits_Q\boldsymbol{F} \cdot \,dA 
\end{equation}
Ecuacion que no esta enumerada
\begin{equation*}
    a=b\quad a=c
\end{equation*}
%-----------------Delimitadores y Llaves----------------------
\section{Delimitadores y llaves}
\begin{equation*}
    \int_{a}^{b}2x\,dx=\left. x^2 \right|_{a}^b
\end{equation*}
\begin{equation*}
    \left [ \frac{2x}{3y} \right]
\end{equation*}
%-----------------Arreglos y Matrices----------------------
\section{Arreglos y Matrices}
\begin{equation*}
    \left[ 
    \begin{array}{cc}
        4&7\\
        3&6
    \end{array}
    \right]
\end{equation*}
\begin{equation*}
    \begin{pmatrix}
        9&8\\
        3&7
    \end{pmatrix}
\end{equation*}
\begin{equation*}
    \begin{pmatrix}
        1 & \ldots & 0 \\
        \vdots & \ddots & \vdots \\
        0 & \ldots & 1
    \end{pmatrix}
\end{equation*}
Función a trozos
\begin{equation*}
    f(x)=
    \left \{
    \begin{array}{lcc}
        x^2 &   si  & x < 2 \\
        x-2 &  si  & x \geq 2
    \end{array}
    \right.
\end{equation*}
M=\bordermatrix{~ & x & y \cr f_1 & 1 & 0 \cr f_2 & 0 &1 \cr}
%---------------Personalización de Expresiones--------------------
\section{Edición de expresiones}
Alineamiento de expresiones
\begin{eqnarray*}
    4(x+y)-2xy &=& 4(x^2+2xy+y^2)-2xy \\
             &=&  4x^2+8xy+4y^2-2xy \\
             &=&  4x^2+4y^2+6xy \\
\end{eqnarray*}
ecuaciones en cajas
\begin{equation*}
    \boxed{x^2+y^2=z^2}
\end{equation*}
Descripciones en ecuaciones.
\[
    z=\overbrace{
        \underbrace{x}_\text{real}+
        \underbrace{yi}_\text{imaginario}
    }^\text{Número complejo}
\]
\section{tips}
$$\vec{\nabla} \cdot \vec {B} = 0$$
\end{document}